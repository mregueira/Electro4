\documentclass[e4_tp2_main.tex]{subfiles}
\begin{document}
\newgeometry{top=2.5cm, bottom=2.0cm, left=2.25cm, right=2.25cm}

\section{Convertidor Boost para Lámpara LED de potencia}

\subsection*{LED's de Potencia: Efecto de la temperatura - Realimentación}
De la hoja de datos de OSRAM para el LUW-W5AP, se da la curva de $\Delta V(T)=V_F-V_F(25^{\circ})$, para la corriente $I_F$ máxima de 1400mA. La misma tiene pendiente negativa, es decir, que a mayor temperatura, la tensión en directa sobre el LED disminuye.

\begin{figure}[H]
\centering
\includegraphics[width=0.3\linewidth]{Imagenes/Punto2/efectoT.png}
\includegraphics[width=0.29\linewidth]{Imagenes/Punto2/IF-VF.png}
\caption{Efecto de la temperatura en la $V_{LED}$: $\Delta V_F(T_j)$ - Curva de $I_F(V_F)$}
\end{figure}

Al trabajar con LED's de potencia, es conveniente realimentar la corriente en lugar de la tensión. Esto se debe a que si se produce una perturbación en la carga (es decir, para el caso de la simulación cortocircuitar dos LED's), si se está regulando tensión, caerá más tensión sobre los LED's restantes, por lo que la corriente aumentará, de acuerdo a la curva de $I_F(V_F)$ provista en la hoja de datos. Esto podría llevar a que los LED's se quemen por exceso de potencia.

\subsection*{LED's de Potencia: Variación del Brillo}
En el realimentador, el amplificador operacional amplifica la tensión sobre la resistencia sensora de la corriente ($R_2$), de manera de obtener a su salida la tensión de referencia de 2.5V para el LT1241. El lazo de realimentación ajusta la corriente para tener en el Pin 2 (FB) dicho valor de tensión dado que el operacional interno a la entrada se encuentra realimentado negativamente de forma externa con un RC entre su salida y el Pin 2 (que es la entrada inversora). Internamente, la entrada no inversora está a un potencial constante de 2.5V. Como el operacional realimentado negativamente busca llegar a que $V^+ = V^-$, de ahí obtenemos que la entrada del Pin 2 se lleva a 2.5V.\par
Teniendo esto en cuenta, Se busca en la hoja de datos el valor de corriente para el cual se obtiene la mitad del brillo (para el valor actual de 2A se obtiene el máximo). De acuerdo a la curva provista:

\begin{figure}[H]
\centering
\includegraphics[width=0.3\linewidth]{Imagenes/Punto2/Lumen-IF.png}
\caption{Curva de $\frac{\Phi_V}{\Phi_V(1.4A)}(I_F)$}
\end{figure}

Se tiene que la mitad del brillo máximo se da a una corriente 
de 700mA. Por lo tanto, la tensión sobre la resistencia sensora $R_2$ será de: 

\[
I_F = 700mA \cdot 0.1\Omega = 0.07V
\]

Sabiendo que a la salida del operacional debe haber 2.5V, se despeja el nuevo valor para $R_6$:
\[
\frac{2.5V}{0.07V} = G = 35.7 \longrightarrow R_6 = 34.7K \Omega
\]

\subsection*{Oscilador - Frecuencia de Switching}
De la hoja de datos, en la sección de Oscilador, se indica que la frecuencia del LT1241 es el doble que la de switching. Entonces, para tener 75KHz, se buscará en la gráfica de $R_TC_T(f)$ el par de valores de componentes acordes para una frecuencia de 150KHz:

\begin{figure}[H]
\centering
\includegraphics[width=0.3\linewidth]{Imagenes/Punto2/RT-OSC.png}
\caption{Curva de $R_TC_T(f)$}
\end{figure}

De donde se obtiene $R_T = 5.3K\Omega$ y $C_T = 2nF$. Es posible verificar mediante las ecuaciones provistas en la misma hoja:

\[
t_r = 0.583 \cdot R_T \cdot C_T \hspace{2cm} t_d = \frac{3.46 \cdot R_T \cdot C_T}{0.0164 \cdot R - 11.73}
\]
\[
T_{OSC} = t_r + t_d \longrightarrow f_{OSC} = 150KHz
\]
\[
f_{SW} = \frac{f_{OSC}}{2} = 75KHz
\]


\end{document}